\documentclass[indentfirst,12pt,letterpaper]{article}
%\documentclass[12pt,letterpaper]{article}

\input psfig.sty
\usepackage[square]{natbib}
\usepackage{ptm}
\usepackage{graphicx}
\usepackage{wrapfig,epsf,boxedminipage}
\usepackage{astjnlabbrev}
\usepackage{amssymb,amsmath}
\usepackage[bf]{caption}
\usepackage{color}


%\usepackage[T1]{fontenc}
%\usepackage[latin1]{inputenc}
%\usepackage{geometry}
%\geometry{verbose,letterpaper}

\setlength{\oddsidemargin}{0.0in}
\setlength{\evensidemargin}{0.0in}
\setlength{\topmargin}{-0.5in}
\setlength{\textheight}{9.0in}
\setlength{\textwidth}{6.5in}
\parindent=15.0pt
\parskip=0pt
%\nofiles
\hyphenation{mag-neto-sphere ion-o-sphere}

\renewcommand{\topfraction}{0.9}
\renewcommand{\bottomfraction}{0.9}
\renewcommand{\floatpagefraction}{0.8}
\def\Evec{{\bf E}}
\def\Bvec{{\bf B}}
\def\vvec{{\bf v}}
\def\Jvec{{\bf J}}

\def\Vvec{{\bf V}}
\def\Pper{P_\perp}
\def\Ppar{P_\parallel}
\def\pper{p_\perp}


%\setlength\parskip{\smallskipamount}
%\setlength\parindent{0pt}

\makeatletter


%%%%%%%%%%%%%%%%%%%%%%%%%%%%%% LyX specific LaTeX commands.
\providecommand{\LyX}{L\kern-.1667em\lower.25em\hbox{Y}\kern-.125emX\@}

\makeatother
\renewcommand*{\contentsname}{}

\begin{document}

%\setcounter{page}{0}

%\thispagestyle{empty}

%\centerline{\bf \Large Global Kinetic Modeling of Alfvenic Aurora}
%\centerline{\bf \Large in Giant Planet Magnetospheres}
%\newpage

\setcounter{page}{0}


\thispagestyle{empty}

%\vspace*{-10ex}
%\tableofcontents
%\addcontentsline{}{References}
%\vspace*{0.4in}

%\noindent References \\
%Biographical Sketches \\
%Current and Pending Support \\
%%Letters of Commitment \\
%Budget Justification \\

%\newpage

\thispagestyle{empty}

%\begin{wrapfigure}{R}{2.3in}
%\centerline{\includegraphics[width=2.3in]{Hill_sketch.pdf}}
%\caption{  }
%\label{Pluto_2d}
%\end{wrapfigure}

%\begin{wrapfigure}[13]{R}{3.3in}
%\centerline{\includegraphics[width=3.3in]{tailrec.pdf}}
%\caption{An illustration of the dominant $B_z$ and small $B_z$
%  (current sheet) regimes that will be addressed in this proposal.}
%\label{ao_sketch}
%\end{wrapfigure}

\centerline{\large Step One proposal: CDAPS - 2014}

\vspace{0.5in}

\centerline{\large \bf The interaction between the Kelvin-Helmholtz instability}\centerline{\large \bf and magnetic reconnection at Saturn's magnetopause boundary}

\vspace{0.5in}
\centerline{Submitted by University of Alaska Fairbanks in support of}
\centerline{NASA ROSES NNH14ZDA001N-CDAPS  }
\centerline{Cassini Data Analysis and Participating Scientists}

\centerline{UAF PI: Peter Delamere}


\newpage

%\section{Scientific objectives and expected significance}

~ {\em At the root of this proposal is understanding how magnetic reconnection operates at Saturn's dayside magnetopause.}
%%%
  The magnetopause boundary is the interface between the solar wind and the magnetosphere, 
  facilitating the transport of mass, momentum, energy, and magnetic flux.  
%%%  
  Key processes mediating the solar wind interaction include magnetic reconnection and Kelvin-Helmholtz (KH) instability.  
%%%
  There are strong evidences suggest that  reconnection must occur.
%%%
  For instance, during the January 2004 Cassini-Hubble Space Telescope campaign, 
  the open flux content of the southern polar region varied between 15 and 50 GWb, 
  with dramatic tail reconnection occurring with high solar wind dynamic pressure.
%%%
   However, to date, there is little {\em in situ} evidence to support a strong reconnection-mediated interaction [\emph{Lai et al.} 2012].
%%%
    \emph{Masters et al.} [2012b] suggested that reconnection at Saturn may be suppressed by large plasma $\beta$
 gradient conditions and this conclusion was supported by \emph{Desroche et al.} [2013].
  
  
 On the other hand, 
 there is evidence indicating that KH modes are operating at Saturn's magnetopause boundary, 
 due, in part, to the rapid planetary rotation ($\sim$10-hour period) and the corotating magnetodisc. 
%%%
 Driven by the large shear flow, 
 KH modes are expected in the pre-noon sector due to the opposing magnetosheath and magnetospheric flows.
%%%
 In the post-noon sector, the flows are both directed tailward and KH is more likely to be stabilized. 
%%%
 However, contrary to expectation, 
\emph{Masters et al.} [2012a] and \emph{Delamere et al.} [2013] showed evidence of KH activity in the post-noon sector.
%%%

 The KH mode can strongly modulate the magnetopause boundary layer,
 and therefore dramatically change the reconnection onset condition [\emph{Otto et al.} 2001].
%%%
In a full three-dimensional geometry, {\emph{Ma et al.}} [2013a, 2013b] have recently demonstrated that
reconnection can trigger KH and conversely KH can trigger reconnection
in the configuration of an anti-parallel magnetic field with a perpendicular shear flow. 
%%%
Therefore, a better understanding of the dawn-dusk asymmetry of KH growth will shed more light on magnetic reconnection at Saturn's magnetopause boundary.
%%%
  {\bf \em  The goal of this proposal is to understand how
  magnetic reconnection operates in Saturn's  large-shear-flow environment.}  We will address the
 following questions:

 {\em
\begin{enumerate}
\item{ How do reconnection and the Kelvin-Helmholtz instability  interact at Saturn's
    magnetopause boundary?}
% and how does plasma $\beta$
 %   influence magnetic reconnection?}
\item{What are the observable signatures of opening reconnection in
    the presence of large shear flows and what is the dependence on
    local time and latitude?}
\item{How do variable solar wind conditions (e.g. magnetic field and
    dynamic pressure) affect reconnection in
    terms of layered magnetosheath structure and in terms of expansion and
    compression of the magnetosphere?}
\end{enumerate}
}

To address these questions we will:

%\begin{enumerate}
(1) Conduct a variety of local 1-D, 2-D, and 3-D MHD, Hall-MHD, and hybrid
simulations of different conditions that are expected for Saturn's
magnetopause boundary.
(2) Validate simulation results with cases of
  large shear flow found at the terrestrial magnetopause.
(3) Compare simulation results with Cassini
  magnetometer (MAG)  and with the Cassini Plasma
  Spectrometer (CAPS) data to identify possible signatures of opening
  magnetic reconnection in the roughly 300 dayside magnetopause boundary crossing between 2004 and 2012.
(4) Provide a comprehensive search in the Cassini data for magnetic
reconnection events and estimate reconnection rates and expected net flux
transport on the dayside magnetopause.







\end{document}




